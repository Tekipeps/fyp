\noindent
This chapter is the last part of the study. It is split into 3 parts: section 5.1 gives a summary of the whole study, section 5.2 gives the conclusion based on the project, and section 5.3 gives suggestions for future study.

\section{Summary}
Heart disease analysis are the most difficult clinical challenges. It is dependent on a thorough examination of the patient's clinical data by clinical specialists. Because of advancements in AI and IT, clinical specialists are eager to create a profoundly accurate, productive, and strong predictive model for heart disease. Data analysis and machine (AI) methodologies were used to assess and summarize the risk of getting heart disease. Expert system technology is also implemented to provide a description of how the patient can improve their heart health based on encoded expert knowledge.

\section{Conclusion}
From this study, I was able to attain my research goals. Heart disease datasets from the Machine Learning Repository have been looked at, cleaned up, and preprocessed to get them ready for the classification process. It has been shown to be a very useful method for auto-tuning parameters and choosing the best set of features that affect the chances of having heart disease. Humans can't be replaced by computers, but by comparing the results of computer-aided diagnosis with pathological findings, doctors can learn more about how to evaluate the areas that computer-aided diagnosis highlights.

\section{Recommendation}
Medical diagnosis is seen as an important but difficult task that needs to be done correctly and efficiently. It would be incredibly advantageous to automate this process. Clinical decisions are often made by doctors based on their gut feelings and years of experience, not on the information-rich data that is hidden in the database. This practice causes biases, mistakes, and high medical costs, all of which hurt the quality of care given to patients. Data mining has the potential to create an environment that is full of knowledge and can help make clinical decisions that are much better.
