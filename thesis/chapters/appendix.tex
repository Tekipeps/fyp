\chapter{Listings}

\begin{lstlisting}[language=Python, caption={Correlation chart}, label={lst:corr}]
	pd.options.display.float_format = "{:,.3f}".format
	plt.figure(figsize=(10,8))
	sns.set_context('notebook',font_scale = 1.15)
	sns.heatmap(df.corr(),annot=True, linewidths=1)
	plt.savefig("corr.png")
	plt.show()
\end{lstlisting}

\begin{lstlisting}[language=Python, caption={Plotting select numerical features with target column}, label={lst:plot-numerical}]
	cols_to_plot = ["Age", "FastingBS"]
	for i in cols_to_plot:
	plt.figure(figsize=(14,5))
	sns.countplot(x=df[i], data=df, hue='HeartDisease')
	plt.legend(['Normal', 'Heart Disease'])
	plt.title(i)
	plt.tight_layout()
\end{lstlisting}

\begin{lstlisting}[language=Python, caption={Plotting categorical values}, label={lst:plot-categorical}]
	# ploting categorical features with target
	for i in categorical:
	plt.figure(figsize=(10,5))
	sns.countplot(x=i, data=df, hue='HeartDisease', edgecolor='black')
	plt.legend(['Normal', 'Heart Disease'])
	plt.title(i)
	plt.show()
\end{lstlisting}

\begin{lstlisting}[numbers=none, label={lst:dataset-info-output}, caption={Dataset information}]
	RangeIndex: 918 entries, 0 to 917
	Data columns (total 12 columns):
	#   Column          Non-Null Count  Dtype  
	---  ------          --------------  -----  
	0   Age             918 non-null    int64  
	1   Sex             918 non-null    object 
	2   ChestPainType   918 non-null    object 
	3   RestingBP       918 non-null    int64  
	4   Cholesterol     918 non-null    int64  
	5   FastingBS       918 non-null    int64  
	6   RestingECG      918 non-null    object 
	7   MaxHR           918 non-null    int64  
	8   ExerciseAngina  918 non-null    object 
	9   Oldpeak         918 non-null    float64
	10  ST_Slope        918 non-null    object 
	11  HeartDisease    918 non-null    int64  
	dtypes: float64(1), int64(6), object(5)
	memory usage: 86.2+ KB
\end{lstlisting}

\begin{lstlisting}[language=Python, caption={Scaling with MinMaxScaler}, label={lst:min-max-scaler}]
	from sklearn.preprocessing import MinMaxScaler
	
	scal = MinMaxScaler()
	X_train = scal.fit_transform(df_nontree[feature_col_nontree].values)
	Y_train = df_nontree[target_col]
	df[string_col]=df[string_col].astype("string")
\end{lstlisting}

\begin{lstlisting}[language=Python, caption={Applying Label Encoder to the dataframe}, label={lst:label-encoding}]
	from sklearn.preprocessing import LabelEncoder
	
	le = LabelEncoder()
	df_cat = df[string_col].apply(le.fit_transform)
\end{lstlisting}
\begin{lstlisting}[language=Python, caption={Applying One-Hot Encoder to the dataframe}, label={lst:one-hot-encoding}]
	from sklearn.preprocessing import OneHotEncoder
	from sklearn.compose import make_column_transformer
	target_col="HeartDisease"
	df_nontree = df.drop(target_col,axis=1)	
	ohe = OneHotEncoder(handle_unknown='ignore')
	transformer = make_column_transformer(
		(OneHotEncoder(), string_col),
			remainder='passthrough',
			verbose_feature_names_out=False
		)
		
	transformed = transformer.fit_transform(df_nontree)
	df_nontree = pd.DataFrame(
		transformed, 
		columns=transformer.get_feature_names_out()
	)
\end{lstlisting}

\begin{lstlisting}[language=Python, caption={App Entrypoint}, label={lst:web-server-entrypoint}]
	from flask import Flask
	from flask_assets import Bundle, Environment
	from app.views.index import index
	from app.config import Config
	
	def create_app():
	app = Flask(__name__)
	app.config.from_object(Config)
	assets = Environment(app)
	
	css = Bundle("src/main.css", output="dist/main.css")
	
	assets.register("css", css)
	css.build()
	
	app.register_blueprint(index)
	
	return app
\end{lstlisting}

\begin{lstlisting}[language=Python, caption={Knowledge base parser}, label={lst:kb-parser}]
	def parse_conditions(kb, table):
		inference = ""
		for attribute in kb:
			attrname = list(attribute.keys())[0]
			for condition in attribute[attrname]["conditions"]:
				if condition.startswith(">"):
					val = int(condition[1:])
					if table[attrname] > val:
						inference += attribute[attrname]["messages"][condition] + " "
				elif condition.startswith("<"):
					val = int(condition[1:])
					if table[attrname] < val:
						inference += attribute[attrname]["messages"][condition] + " "
				elif condition == "True":
					if table[attrname] == True:
						inference += attribute[attrname]["messages"][condition] + " "
				elif condition == "False":
					if table[attrname] == False:
						inference += attribute[attrname]["messages"][condition] + " "
		return inference
\end{lstlisting}