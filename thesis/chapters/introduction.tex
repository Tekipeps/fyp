According to the World Health Organization, 12 million people die each year as a result of coronary disease. Cardiovascular disease is one of the leading causes of mortality and morbidity in the world. Prediction of cardiovascular disease is thought to be one of the most important topic in the data analysis sector. Over the last few years, the overall prevalence of cardiovascular disease has rapidly increased. Many studies have been conducted in an attempt to determine the most significant factors of cardiovascular disease as well as precisely predict the overall risk. Cardiovascular disease is even shown as a quiet executioner, causing the individual's death without any side effects. Early detection of this disease is critical in making decisions about lifestyle adjustments in high-risk patients to reduce repercussions.

\comment{http://forms.gle/EuVegtneuqaP1HgL7}

\comment{
https://archive.ics.uci.edu/ml/datasets/Heart+Disease
https://www.kaggle.com/datasets/cherngs/heart-disease-cleveland-uci
https://www.hindawi.com/journals/mpe/2021/1792201/
https://www.hindawi.com/journals/cin/2022/2973324/
https://ieeexplore.ieee.org/abstract/document/8740989
https://ieeexplore.ieee.org/abstract/document/6164626
http://citeseerx.ist.psu.edu/viewdoc/download?doi=10.1.1.451.9421&rep=rep1&type=pdf
https://www.researchgate.net/profile/V-V-Ramalingam/publication/325116774_Heart_disease_prediction_using_machine_learning_techniques_A_survey/links/5d48560a299bf1995b68266f/Heart-disease-prediction-using-machine-learning-techniques-A-survey.pdf
}

\section{Background of study}
After the cerebrum, the heart is regarded as the second most important organ. Any disruption in the heart causes the entire body to be agitated. Sicknesses, including coronary disease, are the outcome of daily progressions, and it is notable that heart infection is one of the top five causes of illness worldwide. In this sense, anticipating the sickness with perfect time and at the correct moment is critical. Machine learning is an important and necessary cycle in defining and discovering useful data and uncovering hidden patterns in massive data sets.

Information mining and Machine Learning technologies are being used in clinical sciences to solve real-world health challenges by forecasting and diagnosing various illnesses.
Machine learning has been shown to be effective in assisting with decision making and forecasting from the massive amount of data given by the medical industry. This project hopes to predict future cardiovascular disease by analysing patient data that characterizes whether or not they will have cardiovascular disease using machine learning. Machine Learning methods can be of use in this regard. Even though coronary illness can emerge in many forms, there is a standard set of risk variables that influence whether or not someone will eventually be at risk of heart disease. 

We can argue that by obtaining data from various sources, categorizing it under appropriate headings, and examining, to extract the suitable data, this method can be extremely tailored to do the prediction of cardiovascular disease.

\section{Problem statement}
The most challenging task in coronary sickness is recognizing it. There are tools available that can predict heart disease, however they may be expensive or ineffective in determining the risk of coronary sickness in individuals. Early detection of cardiac diseases can reduce death rates and general discomfort. However, screening patients on a regular basis in all instances is absurd, and consulting an expert for 24 hours by a patient is not feasible because it demands more understanding, time, and skill. Because we have so much information in this day and age, we can use various machine learning algorithms to search for hidden patterns in medical data. These hidden patterns can be used for health diagnosis of possible risk of cardiovascular disease.

\section{Aim and objectives}
The aim of this project is:
\begin{itemize}
	\item {to design and execute a heart disease prediction using machine learning techniques that effectively serves both patients and health care centres, particularly in medical institutions.}
\end{itemize}
The objectives of this project include:
\begin{itemize}
	\item {to design and model a heart disease prediction system for hospitals and medical centres.}
	\item {to predict heart disease using various machine learning.}
	\item {to utilize different machine learning calculations to investigate the information for hidden designs.}
\end{itemize}

\comment{The scope of study}
\section{Scope of study}
This study is basically conducted for medical centres. Implementation will be carried out with the health care centre in Anchor University, and may be extended to other medical institutions in Nigeria. 

\comment{It is hoped that the outcome of this research.....}
\section{Significance of study}
The Heart disease machine learning methods software will predict the likelihood of patients getting coronary disease. It enables immense data, for instance, relationships between clinical variables connected with coronary illness and patterns, to be laid out. Also, the software will guarantee the honesty and exactness of the data stored by the authorized users. 

\comment{Chapter organization}
\section{Motivation}
The primary innovation in completing this work is to present a heart disease prediction model to estimate the occurrence of heart sickness. Furthermore, the goal of this research is to identify the optimal arrangement estimation for detecting the possibility of heart disease in a patient. This work is genuine since it does a relative report and analysis using three characterization algorithms, specifically Decision Tree, Naive Bayes, and Random Forest, at various levels of assessment. Although these are commonly used machine learning methods, heart disease prediction is a critical task requiring the highest possible precision. As a result, the three algorithms are evaluated at various to determine the best in predicting coronary disease.

\section{Factors of heart disease}

The factors that cause the heart to break down include any major abnormal heart or blood vessel condition known as coronary disease. The different sorts of heart disease are:

\begin{itemize}
	\item{Coronary Artery Disease: refers to arrangement of cholesterol plaque which causes solidifying or restricting of the heart channel, (which supplies blood to the heart).}
	\item{Cardiomyopathy: refers to a disease of the heart muscle that makes it harder for the heart to pump blood to the rest of the body.}
	\item{Angina: a type of chest pain caused by reduced blood flow to the heart.}
	\item{Valvular Heart Disease: refers to damage or decease on at least one of the four valves of the heart}
	\item{Congenital Heart Disease: refers to a variety of birth defects that affect the normal functioning of the heart.}
	\item{Cerebrovascular Disease: a set of conditions that affect blood flow and the blood vessels in the brain.}
	\item{Heart attack: A heart attack occurs when an artery carrying blood and oxygen to the heart becomes blocked.}
	\item {Heart failure: Heart failure occurs when the heart muscle fails to adequately pump blood.}
	\item{Rheumatic Heart Disease: a condition where the heart muscles and valves are permanently damaged due to rheumatic fever}
\end{itemize}

The above-mentioned heart diseases are only a few of the many heart diseases that affect both men and women. These can be detected by paying close attention to the common factors that cause heart disease. The following are some of the most prevalent symptoms of heart disease:

\begin{itemize}
	\item{Having pain in the upper body such as the arms, jaw, neck, back or upper stomach.}
	\item{Significant strain, discomfort, or a burning pain in the chest.}
	\item{Rapidly expanding heartbeat.}
	\item{Unsteadiness, sweating and nausea.}
	\item{Swelling, anxiety, cough, aching, burning and so on...}
\end{itemize}

\section{Limitations of study}
There are few drawbacks to be aware of. These include:
\begin{itemize}
	\item{At the first release, the heart disease prediction may not have all the features desired by medical practitioners.}
	\item{Due to time constraints, the first release may not have all intended functionalities, but would be improved upon in the nearest future.}
\end{itemize}