\comment{http://forms.gle/EuVegtneuqaP1HgL7}

\comment{
https://archive.ics.uci.edu/ml/datasets/Heart+Disease
https://www.kaggle.com/datasets/cherngs/heart-disease-cleveland-uci
https://www.hindawi.com/journals/mpe/2021/1792201/
https://www.hindawi.com/journals/cin/2022/2973324/
https://ieeexplore.ieee.org/abstract/document/8740989
https://ieeexplore.ieee.org/abstract/document/6164626
http://citeseerx.ist.psu.edu/viewdoc/download?doi=10.1.1.451.9421&rep=rep1&type=pdf
https://www.researchgate.net/profile/V-V-Ramalingam/publication/325116774_Heart_disease_prediction_using_machine_learning_techniques_A_survey/links/5d48560a299bf1995b68266f/Heart-disease-prediction-using-machine-learning-techniques-A-survey.pdf
}

\section{Background of study}
According to \citefullauthor{2021Cardiovascular}, cardiovascular disease also known as heart disease is one of the leading causes of mortality and morbidity in the world. Over 12 million people die each year as a result of heart disease. Any significant, abnormal condition of the heart or blood vessels is referred to as cardiovascular disease (CVD) (arteries, veins). Coronary heart disease (CHD), stroke, peripheral artery disease, congenital heart disease, endocarditis, and a variety of other disorders are all examples of cardiovascular diseases.

The heart is regarded as one important organ in the body. Any disruption in the heart causes the entire body to be agitated. Sicknesses, including heart disease, are the outcome of daily progressions, and it is notable that heart disease is one of the top five causes of illness worldwide \citep{2021Cardiovascular}. Many cardiovascular problems can be avoided. Over the last few years, the overall prevalence of cardiovascular disease has rapidly increased. Many studies have been conducted in an attempt to determine the most significant factors of cardiovascular disease as well as precisely predict the overall risk. Cardiovascular disease is even shown as a quiet executioner, causing the individual's death without any side effects. Early detection of this disease is critical in making decisions about lifestyle adjustments in high-risk patients to reduce repercussions. In this sense, anticipating heart disease with perfect time and at the correct moment is critical. Machine learning is an important and necessary cycle in defining and discovering useful data and uncovering hidden patterns in massive datasets.

Information mining and Machine Learning technologies are being used in clinical sciences to solve real-world health challenges by forecasting and diagnosing various illnesses.
Machine learning has been shown to be effective in assisting with decision making and forecasting from the massive amount of data given by the medical industry. This project hopes to predict the risk of heart disease by analysing patient data that characterizes whether or not they will have cardiovascular disease using machine learning algorithms. Even though heart disease can emerge in many forms, there is a standard set of risk variables that influence whether or not someone will eventually be at risk of heart disease. 


\section{Problem statement}
The most challenging task in coronary sickness is recognizing it. There are tools available that can predict heart disease, however they may be expensive or ineffective in determining the risk of coronary sickness in individuals. Early detection of heart diseases can reduce death rates and general discomfort. However, screening patients on a regular basis in all instances is absurd, and consulting an expert for 24 hours by a patient is not feasible because it demands more understanding, time, and skill. Because we have so much information in this day and age, we can use various machine learning algorithms to search for hidden patterns in medical data. These hidden patterns can be used for health diagnosis of possible risk of heart disease.

\section{Significance of study}
The Heart disease machine learning software will predict the likelihood of patients getting heart disease. It enables immense data, for instance, relationships between clinical variables connected with heart disease and patterns, to be laid out. Also, the software will provide an inference based off the medical data inputted by its users. The proposed system will help users, researchers and medical practitioners to know if an individual is likely to have heart disease based off their medical data.

\comment{Chapter organization}
\section{Motivation}
The key contribution that this work makes is the presentation of a heart disease prediction system, which can assess the likelihood of an individual developing heart disease. In addition to that, the purpose of this research is to determining whether or not a patient is at risk of having heart disease and provide some expert knowledge based on the user's medical data. This work is genuine since it does a relative report and analysis using four classification algorithms, specifically Decision Tree, Naive Bayes, Logistic regression and K-nearest neighbours, at various levels of assessment. Although these are commonly used machine learning methods, heart disease prediction is a critical task requiring the highest possible precision. As a result, the four algorithms are evaluated to determine the best in predicting heart disease. Some of the reviewed studies includes
\citefullauthor{waghulde2014genetic} research on heart disease which led them to collect information about multi-facet neutral networks and back spread learning calculations for network preparation. \cite{srinivas2010applications} suggested that data mining strategies be used to predict cardiovascular failure. He focused on arrangement calculations, such as Decision tree and Naive Bayes, to tell if a stroke infection was present. \cite{pattekari2012prediction} used the Naive Bayesian Classification procedure to help people make decisions in the coronary illness expectation framework. In the end, the problem with all of these projects is that they did not design a system that allows users easy access to use the models. Also, the proposed systems were limited such that they did not provide a way for users to have more knowledge about the inputted health data.

\section{Aim and objectives}
The aim of this project is to develop a heart disease prediction system using machine learning techniques that effectively serves both patients and health care centres, particularly in medical institutions, and to build a submodule that provides inference (expert knowledge) based on user's medical data and prediction output. 

The objectives of the study are to:
\begin{enumerate}[label=(\alph*)]
	\item  design  of a heart disease prediction model for hospitals and medical centres
	\item an inference engine that provides inference based on the prediction output and information provided by the patient
\end{enumerate}

\section{Methodology}
The methods used in the study include creating machine learning models using classification algorithms to predict the risk of heart disease. These models are trained, tested and saved for in in a web application. The web app provides an interface for users to interact with the machine learning model. After prediction by the model, an inference engine produces expert knowledge about the health data from rules encoded in a knowledge base. More detailed information about the methodology will be discussed in \hyperref[chap:3]{Chapter 3}

\comment{The scope of study}
\section{Scope of study}
The primary audience for this study is researchers, hospitals and medical clinics. The implementation will begin with the health care centre that is located within Anchor University, and it is possible that it may be expanded to include additional medical institutions located within Nigeria.


\section{Limitations of study}
There are few drawbacks to be aware of. These include:

At the first release, the heart disease prediction may not have all the features desired by medical practitioners.

Due to time constraints, the first release may not have all intended functionalities like a well formed knowledge base, but would be improved upon in the nearest future.
